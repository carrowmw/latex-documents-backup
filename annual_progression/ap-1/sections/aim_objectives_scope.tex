\section{Aim, Objectives and Scope}

\subsection{Aim}
To develop an AI system for the prediction of spatiotemporal dynamics in the built environment using near real-time geospatial IoT sensor data.


\subsection{Research Objectives}
\begin{itemize}
    \item RO1: Develop tools to assess the quality of near real-time sensor data.
    \item RO2: Assess the spatiotemporal dependency of near real-time sensor data.
    \item RO3: Assimilate outputs of agent-based models with real-time sensor data to monitor urban systems in real-time.
    \item RO4: Evaluate the approach using real-world case-studies and develop a roadmap for scalable deployment.
\end{itemize}


\subsection{Scope}
By applying existing data-quality frameworks to develop a data quality monitoring system RO1 seeks to quantify data quality in a manner that can both be interpreted by an end-user (dashboards) and a programme interface for use in later objectives. Making predictions requires an understanding of the dependency of the sensors in time and space – how strongly do sensor measurements relate to the wider sensor network – a spatial AI model in combination with existing knowledge about agent behaviour will be investigated (RO2). To enhance prediction, a data assimilation approach using the outputs of much larger computational models (such as an agent-based model of transport demand) will be investigated (RO3). Ensuring this research has real-world applicability and elevating the technology to a level that meets the needs of its users, a roadmap will be developed. A variety of computational models and sensor network combinations will be investigated to achieve this (RO4).

\subsection{Deliverables}
The research aims to contribute the following to the field of complex systems modelling:

\begin{itemize}
    \item Greater understanding about the real dynamics of complex urban systems.
    \item An enhanced understanding of causality in complex urban systems.
\end{itemize}

And to deliver the following technical capabilities:

\begin{itemize}
    \item A demonstration of value for the data collected by centralised urban repositories.
    \item Pathways to making this data ‘AI ready’.
    \item A demonstration of a real-time cloud-based web-app that provides useful information derived from the sensor data that can be used for improved decision making.
    \item A package of code that conforms to best-practice and is built to be compatible with urban digital-twin frameworks such as DAFNI/Gemini.
    \item A series of publications showcasing any scientific advancements made by this research.
\end{itemize}