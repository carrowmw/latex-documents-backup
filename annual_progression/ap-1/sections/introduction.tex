\section{Introduction}

Predicting short-term behaviour of agents within a city requires reliable high-quality data, real-time processing infrastructure, and models that are capable of generating accurate predictions about complex situations at speed. Enabling this technology could have profound benefits for cities in domains such as emergency response, resource optimisation and decision-making, thus allowing urban areas to better to respond to unsafe situations such as overcrowding or flooding.

Centralised data repositories are appearing in cities around the world, collecting near real-time data from distributed sensor networks. However, issues with data quality need to be addressed before predictive systems can be developed. The volume and velocity of the data collected by these sensors requires automated quality-aware systems to assess veracity before the data can be incorporated into the decision-making systems enabled by short-term predictive capabilities.

Whilst there is existing work in this area but it has not yet been applied at scale to data collected by urban sensor networks.

